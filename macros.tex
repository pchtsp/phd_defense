%& BEAMER

\beamertemplatenavigationsymbolsempty
\setbeamertemplate{footline}[frame number]

\newcommand{\sectiontitle}{}
%\newcommand{\sectionnumber}{}
\newcommand{\newsection}[1]{\section{#1}\renewcommand{\sectiontitle}{#1}}

\newcommand{\mdpgray}{black!10}

\newcommand{\citesize}{\tiny} % \scriptsize

\newcommand{\ftitleintro}[1]{%
	\frametitle{#1}%
}%
\newcommand{\ftitlei}[1]{%
	\frametitle{\makebox[1.13\textwidth]{#1\hfill\includegraphics[height=5mm, trim={0mm 2mm 0mm 0mm}]{images/numbers/i}}}%
}%
\newcommand{\ftitleii}[1]{%
	\frametitle{\makebox[1.13\textwidth]{#1\hfill\includegraphics[height=5mm, trim={0mm 2mm 0mm 0mm}]{images/numbers/ii}}}%
}%
\newcommand{\ftitleiii}[1]{%
	\frametitle{\makebox[1.13\textwidth]{#1\hfill\includegraphics[height=5mm, trim={0mm 2mm 0mm 0mm}]{images/numbers/iii}}}%
}%

% Contributions drawing
\newcommand{\cla}{red}
\newcommand{\clb}{cyan}
\newcommand{\clc}{orange}
\newcommand{\arrowcolor}{orange!30}
\newcommand{\thk}{thick}
\newcommand{\cone}[3]{
	\draw [draw=none, fill=#3, opacity=0.25] plot [] coordinates {(#1, #2) (#1 + 9mm, #2 + 8mm) (#1 + 9mm, #2 - 8mm)};
	
%	\draw[dashed] (#1, #2) -- (#1 + 9mm, #2 + 8mm);
%	\draw[dashed] (#1, #2) -- (#1 + 9mm, #2 - 8mm);
	
	\draw (#1, #2) node [oltadot, color=\clb, fill=white] {};
}
\newcommand{\finalcone}[3]{
	\draw [draw=none, fill=#3, opacity=0.25] plot [] coordinates {(#1, #2) (#1 + 9mm, #2 + 8mm) (#1 + 9mm, #2 - 8mm)};
	
	\draw (#1, #2) node [oltadot, color=\clb, fill=white] {};
}
\newcommand{\lipub}[4]{ % color thickness x y
	\draw [#1, #2] (#3, #4) -- (#3-1, #4+2.8);
	\draw [#1, #2] (#3, #4) -- (#3+1, #4+2.8);
}
\newcommand{\lipcone}[4]{ % color thickness x y fillcolor
	\def\ratio{0.35}
	\draw [#1, #2, dashed, name path = A] (#3-\ratio*1, #4-\ratio*2.8) -- (#3+\ratio*1, #4+\ratio*2.8);
	\draw [#1, #2, dashed, name path = B] (#3+\ratio*1, #4-\ratio*2.8) -- (#3-\ratio*1, #4+\ratio*2.8);
	\tikzfillbetween[of = A and B]{#1, opacity=0.2};
}


%\begin{tcolorbox}[title=Planification]
%	TODO
%\end{tcolorbox}

\newtcolorbox{mdpbox}[1][]{
	notitle,
	height=1.3cm,
	colframe=green!60!black,
	colback=green!20!white,
	boxrule=-0.1mm,
	valign=center,
	valign lower=center,
	halign=center,
	#1
}

%\newcommand{\sectiontitle}{#1}
%\newcommand{\setsectiontitle}[1]{\renewcommand{sectiontitle}{#1}}

%& TEXTS
\newcommand{\ie}{\emph{i.e.}}
\newcommand{\eg}{\emph{e.g.}}
\newcommand{\vs}{\emph{vs.}}
\newcommand{\etc}{\emph{etc.}}
\newcommand{\wrt}{with respect to}
\newcommand{\st}{s.t.}
\newcommand{\ub}{upper bound}
\newcommand{\Ub}{Upper bound}
\newcommand{\cf}{closed-form}
\newcommand{\fp}{fixed-point}

\newcommand{\Qfun}{Q-function}
\newcommand{\Qval}{Q-value}

\newcommand{\ts}{tree search}
\newcommand{\Ts}{Tree search}
\newcommand{\TS}{Tree Search}
\newcommand{\ol}{open-loop}
\newcommand{\Ol}{Open-loop}
\newcommand{\cl}{closed-loop}
\newcommand{\Cl}{Closed-loop}

\newcommand{\wc}{worst case}
\newcommand{\Wc}{Worst case}
\newcommand{\ra}{risk averse}
\newcommand{\Ra}{Risk averse}

\newcommand{\ns}{non-stationary}
\newcommand{\Ns}{Non-stationary}
\newcommand{\NS}{Non-Stationary}
\newcommand{\nsty}{non-stationarity}
\newcommand{\nsties}{non-stationarities}
\newcommand{\Nsty}{Non-stationarity}
\renewcommand{\ne}{non-episodic}

\renewcommand{\l}{lifelong}
\renewcommand{\ll}{\l{} learning}
\newcommand{\lml}{\l{} {ML}}
\newcommand{\lrl}{\l{} {RL}}
\newcommand{\Lrl}{Lifelong {RL}}

\newcommand{\perspectiveintrosentence}{The content of this chapter suggests a few perspectives.}

\newcommand{\proofsentence}[2]{The proof of #1 is reported in the Appendix, Chapter~\ref{chap:app:proofs}, Section~\ref{#2}.}

\newcommand{\proofsentenceplural}[2]{The proofs of #1 are reported in the Appendix, Chapter~\ref{chap:app:proofs}, Section~\ref{#2}.}

%& Notations for larger symbols
\makeatletter
\newcommand{\vast}{\bBigg@{3}}
\newcommand{\Vast}{\bBigg@{5}}
\newcommand{\vastl}{\mathopen\vast}
\newcommand{\vastm}{\mathrel\vast}
\newcommand{\vastr}{\mathclose\vast}
\newcommand{\Vastl}{\mathopen\Vast}
\newcommand{\Vastm}{\mathrel\Vast}
\newcommand{\Vastr}{\mathclose\Vast}
\makeatother

%& MATHS
\newcommand*{\SET}[1]{\ensuremath{ \left\{ #1 \right\} }}
\newcommand*{\EXP}[1]{\ensuremath{ \mathbb{E} \left( #1 \right) }}
\newcommand*{\EXPLOW}[1]{\ensuremath{ \mathbb{E}_{#1} }}
\newcommand*{\EXPENS}[2]{\ensuremath{ \mathbb{E}_{#1} \left( #2 \right) }}
\newcommand{\argmax}{\operatornamewithlimits{argmax}}
\newcommand{\argmin}{\operatornamewithlimits{argmin}}
\newcommand{\MAX}[1]{\ensuremath{ \max \left\{ #1 \right\} }}
\newcommand{\MAXENS}[2]{\ensuremath{ \max_{#1} \left\{ #2 \right\} }}
\newcommand{\MIN}[1]{\ensuremath{ \min \left\{ #1 \right\} }}
\newcommand{\MINENS}[2]{\ensuremath{ \min_{#1} \left\{ #2 \right\} }}
\newcommand{\ARGMAX}[1]{\ensuremath{ \argmax \left\{ #1 \right\} }}
\newcommand{\ARGMAXENS}[2]{\ensuremath{ \argmax_{#1} \left\{ #2 \right\} }}
\newcommand{\ARGMIN}[1]{\ensuremath{ \argmin \left\{ #1 \right\} }}
\newcommand{\ARGMINENS}[2]{\ensuremath{ \argmin_{#1} \left\{ #2 \right\} }}
\DeclarePairedDelimiter{\ceil}{\lceil}{\rceil}
\newcommand{\bigo}{\ensuremath{\mathcal{O}}} % big O
\newcommand{\bigotilde}{\ensuremath{\tilde{\mathcal{O}}}}
\newcommand{\functionspace}[2]{\mathcal{F} \left( #1, #2 \right)}
\newcommand{\FUNCTION}[5]{
	\ensuremath{
		\begin{array}{llll}
			#1: & #2 & \rightarrow & #3 \\
			& #4 & \mapsto & #5
		\end{array}
	}
}

% Norms
\renewcommand{\L}{\ensuremath{\mathcal{L}}}
\newcommand{\absnorm}[1]{\ensuremath{ \left| #1 \right| }}
\newcommand{\nnorm}[2]{\ensuremath{ \left\lVert #1 \right\rVert_{#2} }}
\newcommand{\onenorm}[1]{\nnorm{#1}{1}}
\newcommand{\twonorm}[1]{\nnorm{#1}{2}}
\newcommand{\inftynorm}[1]{\nnorm{#1}{\infty}}

% Sets
\newcommand{\R}{\ensuremath{\mathbb{R}}}
\newcommand{\Z}{\ensuremath{\mathbb{Z}}}
\newcommand{\N}{\ensuremath{\mathbb{N}}}
\newcommand{\M}{\ensuremath{\mathcal{M}}}
\newcommand{\T}{\ensuremath{\mathcal{T}}}
\newcommand{\B}{\ensuremath{\mathcal{B}}}
\newcommand{\ball}[3]{\ensuremath{\B_{#1} \left( #2, #3 \right)}} % metric center radius
\newcommand{\intrange}[2]{\ensuremath{\left\{#1, \dots, #2\right\}}} % range of integers
\newcommand{\intrangeinfty}[2]{\ensuremath{\left\{#1, #2, \dots \right\}}} % infinite range of integers
\newcommand{\tuple}[2]{\ensuremath{\left( #1, #2 \right)}} % 2-uple
\newcommand{\threeuple}[3]{\ensuremath{\left( #1, #2, #3 \right)}} % 3-uple
\newcommand{\fouruple}[4]{\ensuremath{\left( #1, #2, #3, #4 \right)}} % 4-uple
\newcommand{\setcomplement}[1]{\ensuremath{{#1}^{c}}}

% Probabilities
\renewcommand{\Pr}{\ensuremath{\textbf{Pr}}}
\newcommand{\condPr}[2]{\ensuremath{\Pr \left( #1 \;\middle|\; #2 \right)}}
\renewcommand{\P}{\ensuremath{\mathcal{P}}} % set of probabilities

% Display
\newcommand{\mthspc}{\ensuremath{\,}}
\newcommand{\ms}{\ensuremath{\,}}
\newcommand{\eqspacing}{\ensuremath{\;}}
\newcommand{\rforall}{\ensuremath{\eqspacing\forall}}
\newcommand{\phleq}{\hphantom{\leq\text{ }}}
\newcommand{\phgeq}{\hphantom{\geq\text{ }}}
\newcommand{\pheq}{\hphantom{=\text{ }}}

%& MDP elements
\newcommand{\initstatedistrib}{\ensuremath{\P_0}}
\renewcommand{\S}{\ensuremath{\mathcal{S}}}
\newcommand{\A}{\ensuremath{\mathcal{A}}}
\newcommand{\cardinalactions}{\ensuremath{\SET{\text{Right, Up, Left, Down}}}}
\newcommand{\sa}{\ensuremath{\tuple{s}{a}}}
\newcommand{\SA}{\ensuremath{\S \times \A}}
\newcommand{\STA}{\ensuremath{\S \times \T \times \A}}
\newcommand{\nS}{\ensuremath{S}} % number of states
\newcommand{\nA}{\ensuremath{A}} % number of actions

\newcommand{\transition}{\ensuremath{T}}
\newcommand{\tra}[3]{\ensuremath{\transition{}_{#1 #3}^{#2}}} % s a s'
\newcommand{\trahat}[3]{\ensuremath{\hat{\transition{}}_{#1 #3}^{#2}}}
\newcommand{\trabar}[3]{\ensuremath{\bar{\transition{}}_{#1 #3}^{#2}}}
\newcommand{\trat}[4]{\ensuremath{\transition{}_{#2} \left( #4 \;\middle|\; #1, #3 \right)}} % s t a s'
\newcommand{\trahatt}[4]{\ensuremath{\hat{\transition{}}_{#2} \left( #4 \;\middle|\; #1, #3 \right)}}
\newcommand{\trabart}[4]{\ensuremath{\bar{\transition{}}_{#2} \left( #4 \;\middle|\; #1, #3 \right)}}
\newcommand{\tratildet}[4]{\ensuremath{\tilde{\transition{}}_{#2} \left( #4 \;\middle|\; #1, #3 \right)}}

\newcommand{\reward}{\ensuremath{r}}
\newcommand{\rew}[3]{\ensuremath{\reward{}_{#1 #3}^{#2}}} % s a s'
\newcommand{\rewhat}[3]{\ensuremath{\hat{\reward{}}_{#1 #3}^{#2}}}
\newcommand{\rewbar}[3]{\ensuremath{\bar{\reward{}}_{#1 #3}^{#2}}}
\newcommand{\rewhatbar}[3]{\ensuremath{\hat{\bar{\reward{}}}_{#1 #3}^{#2}}}
\newcommand{\rewt}[4]{\ensuremath{\reward{}_{#2} \left( #1, #3, #4 \right)}} % s t a s'
\newcommand{\rewhatt}[4]{\ensuremath{\hat{\reward{}}_{#2} \left( #1, #3, #4 \right)}}
\newcommand{\rewbart}[4]{\ensuremath{\bar{\reward{}}_{#2} \left( #1, #3, #4 \right)}}
\newcommand{\rewtildet}[4]{\ensuremath{\tilde{\reward{}}_{#2} \left( #1, #3, #4 \right)}}

\newcommand{\Reward}{\ensuremath{R}}
\newcommand{\Rew}[2]{\ensuremath{\Reward{}_{#1}^{#2}}} % s a
\newcommand{\Rewhat}[2]{\ensuremath{\hat{\Reward{}}_{#1}^{#2}}}
\newcommand{\Rewbar}[2]{\ensuremath{\bar{\Reward{}}_{#1}^{#2}}}
\newcommand{\Rewhatbar}[2]{\ensuremath{\hat{\bar{\Reward{}}}_{#1}^{#2}}}
\newcommand{\Rewt}[3]{\ensuremath{\Reward{}_{#2} \left( #1, #3 \right)}} % s t a
\newcommand{\Rewhatt}[3]{\ensuremath{\hat{\Reward{}}_{#2} \left( #1, #3 \right)}}
\newcommand{\Rewbart}[3]{\ensuremath{\bar{\Reward{}}_{#2} \left( #1, #3 \right)}}
\newcommand{\Rewtildet}[3]{\ensuremath{\tilde{\Reward{}}_{#2} \left( #1, #3 \right)}}

\newcommand{\Rmax}{\ensuremath{\Reward{}_{\text{max}}}}
\newcommand{\Vmax}{\ensuremath{V_{\text{max}}}}
\newcommand{\return}{\ensuremath{Z}}
\newcommand{\returnn}[1]{\ensuremath{\return^{#1}}}
\newcommand{\returnmax}{\ensuremath{\return_{\text{max}}}}
\newcommand{\horizon}{\ensuremath{H}}

%& Value functions
\newcommand{\V}[2]{\ensuremath{V^{#1}_{#2}}} % \pi M
\newcommand{\Vhat}[2]{\ensuremath{\hat{V}^{#1}_{#2}}} % \pi M
\newcommand{\Q}[2]{\ensuremath{Q^{#1}_{#2}}} % \pi M
\newcommand{\Qhat}[2]{\ensuremath{\hat{Q}^{#1}_{#2}}} % \pi M
\newcommand{\bellmanoperator}[2]{\ensuremath{\mathscr{T}^{#1}_{#2}}} % Bellman operator {\pi}{V}

% NS value functions
\newcommand{\Vt}[3]{\ensuremath{V^{#1}_{#2 #3}}} % \pi t M
\newcommand{\Qt}[3]{\ensuremath{Q^{#1}_{#2 #3}}} % \pi t M

% \Wc{} NS value functions
\newcommand{\Vwc}[3]{\ensuremath{\bar{V}^{#1}_{#2, #3}}} % \pi t_0 t
\newcommand{\Qwc}[3]{\ensuremath{\bar{Q}^{#1}_{#2, #3}}} % \pi t_0 t
\newcommand{\Vwcapprox}[3]{\ensuremath{\hat{V}^{#1}_{#2, #3}}} % \pi t_0 t
\newcommand{\Qwcapprox}[3]{\ensuremath{\hat{Q}^{#1}_{#2, #3}}} % \pi t_0 t

%& Algorithms / policies
\newcommand{\algo}{\ensuremath{\mathscr{A}}}
\newcommand{\randompolicy}{\ensuremath{\pi_{\text{random}}}}
\newcommand{\treepolicy}{\ensuremath{\pi_{\text{tree}}}}
\newcommand{\ucttreepolicy}{\ensuremath{\pi_{\text{UCT}}}}
\newcommand{\defaultpolicy}{\ensuremath{\pi_{\text{default}}}}

\newcommand{\qlearningalg}{Q-Learning}
\newcommand{\rmaxalg}{RMAX}
\newcommand{\maxqinitalg}{MaxQInit}
\newcommand{\lrmaxalg}{{LRMAX}}
\newcommand{\lrmaxalgs}{{LRMAX}}
\newcommand{\lrmaxalgl}{{LRMAX}}
\newcommand{\lrmaxqinitalg}{LRMaxQInit}
\newcommand{\mctsalg}{{MCTS}}
\newcommand{\mctsalgs}{{MCTS}}
\newcommand{\uctalg}{{UCT}}
\newcommand{\oluctalg}{{OLUCT}}
\newcommand{\oluctalgs}{{OLUCT}}
\newcommand{\oltaalg}{{OLTA}}
\newcommand{\oltaalgs}{{OLTA}}
\newcommand{\oltaalgl}{{OLTA}}
\newcommand{\olopalg}{{OLOP}}
\newcommand{\holopalg}{{HOLOP}}
\newcommand{\oletsalg}{{OLETS}}
\def\algocommentcolor{green!50!black} % Comment color in algorithms
\def\algocomment#1{
	 %\hfill
	 \texttt{\footnotesize \color{\algocommentcolor}\# #1}
}
% TODO Check whether algorithms names are included in list of acronyms

%& Trees (MCTS UCT OLUCT OLTA Minimax and so on)
\newcommand{\budget}{\ensuremath{B}} % tree budget
\newcommand{\node}{\ensuremath{\nu}}
\newcommand{\chancenode}[2]{\ensuremath{\node_{#1}^{#2}}} % {s}{a}
\newcommand{\decisionnode}[1]{\ensuremath{\node_{#1}}} % {s}

%& Chapter 2 (OLTA)
\newcommand{\nvisits}[3]{\ensuremath{T^{#1}_{#2,#3}}} % {depth}{action index}{decision epoch}
\newcommand{\returnestimate}{\ensuremath{\hat{\return}}}
\newcommand{\returnestimateat}[3]{\ensuremath{\returnestimate^{#1}_{#2, #3}}} % {depth}{action index}{decision epoch}
\newcommand{\tree}{\ensuremath{\Gamma}} % tree
\renewcommand{\a}[1]{\ensuremath{a^{#1}}} % ith action of \A

%& Chapter 3 (RATS)
\newcommand{\wass}[1]{\ensuremath{W_{#1}}} % p
\newcommand{\wasserstein}[3]{\ensuremath{\wass{#1} \left( #2, #3 \right)}} % p mu nu
\newcommand{\snapshot}[1]{\ensuremath{\text{MDP}_{#1}}} % t
\newcommand{\dmax}{\ensuremath{d_{\text{max}}}} % maximum depth
\newcommand{\heuristicfunction}{\ensuremath{\mathcal{H}}} % heuristic function
\newcommand{\heuristicerror}{\ensuremath{\delta_{\heuristicfunction}}} % heuristic error

%& Chapter 4 (LLRL)
\newcommand{\agent}{\ensuremath{\blacktriangle}}
\newcommand{\model}{\ensuremath{M}}
\newcommand{\lrldistrib}{\ensuremath{\mathcal{D}}}
\newcommand{\nknown}{\ensuremath{n_{\text{known}}}}
\newcommand{\ntimesteps}{\ensuremath{\tau}}
\newcommand{\prior}{\ensuremath{D_{\text{max}}}} % prior knowledge
\newcommand{\pmin}{\ensuremath{p_{\min}}}

% former
%\newcommand{\Dmodel}[4]{\ensuremath{D_{#3}^{#1 #2} #4}} % {M}{\bar{M}}{f}{(s, a)}
%\newcommand{\Dmodelhat}[3]{\ensuremath{\hat{D}^{#1 #2} #3}}
%\newcommand{\Deltamodel}[3]{\ensuremath{\Delta^{#1 #2}(#3)}} % M1 M2 s,a

% new
\newcommand{\modpm}[5]{\ensuremath{D_{#1 #2}^{#3} ( #4, #5 )}} % {s}{a}{f}{M1}{M2}
\newcommand{\modiv}[5]{\ensuremath{D_{#1 #2} ( #4 \| #5 ) }} % {s}{a}{f}{M1}{M2} -> light notation for \modpm when f = gamma times the optimal value function of {M2}
\newcommand{\modivhat}[4]{\ensuremath{\hat{D}_{#1 #2} ( #3 \| #4 )}} % {s}{a}{M1}{M2}

\newcommand{\mdpdiv}[4]{\ensuremath{d_{#1 #2} ( #3 \| #4 ) }} % {s}{a}{M1}{M2}
\newcommand{\mdpdivhat}[4]{\ensuremath{\hat{d}_{#1 #2} ( #3 \| #4 ) }} % {s}{a}{M1}{M2}
\newcommand{\mdppm}[4]{\ensuremath{\Delta_{#1 #2} ( #3, #4 ) }} % {s}{a}{M1}{M2}

% additional notations for a policy pi
\newcommand{\mdpdivpi}[5]{\ensuremath{d_{#1 #2}^{#3} ( #4 \| #5 ) }} % {s}{a}{pi}{M1}{M2}
\newcommand{\mdppmpi}[5]{\ensuremath{\Delta_{#1 #2}^{#3} ( #4, #5 ) }} % {s}{a}{pi}{M1}{M2}

% additional notations for DP sequences
\newcommand{\mdpdivn}[5]{\ensuremath{d_{#1 #2}^{#3} ( #4 \| #5 ) }} % {s}{a}{n}{M1}{M2}
\newcommand{\mdpdivhatn}[5]{\ensuremath{\hat{d}_{#1 #2}^{#3} ( #4 \| #5 ) }} % {s}{a}{n}{M1}{M2}

%& Definitions / allocation
\newcommand{\receives}{\ensuremath{\leftarrow}}
\newcommand{\eqdef}{\ensuremath{\triangleq}}
%\newcommand{\indef}{\ensuremath{\stackrel{\mathclap{\normalfont\mbox{\footnotesize def}}}{\in}}}
\newcommand{\indef}{\ensuremath{\stackrel{\mathclap{\scriptscriptstyle \bigtriangleup}}{\in}}}
\newcommand*{\qm}[1]{``#1''}

%& Rename natbib commands
%\newcommand{\citet}[1]{\citeauthor{#1} \shortcite{#1}}
%\newcommand{\citet}[1]{\cite{#1}} % TODO retablir
%\newcommand{\citep}[1]{(\cite{#1})}

%& Theorems definition
%\newtheoremstyle{stylename}% name of the style to be used
%	{spaceabove}% measure of space to leave above the theorem. E.g.: 3pt
%	{spacebelow}% measure of space to leave below the theorem. E.g.: 3pt
%	{bodyfont}% name of font to use in the body of the theorem
%	{indent}% measure of space to indent
%	{headfont}% name of head font
%	{headpunctuation}% punctuation between head and body
%	{headspace}% space after theorem head; " " = normal interword space
%	{headspec}% Manually specify head

\newtheoremstyle{break}
	{11pt}{11pt}%
	{\itshape}{}%
	{\bfseries}{}%
	{\newline}{}%
\theoremstyle{break}

\theoremstyle{plain}
%\newtheorem{theorem}{Theorem}[chapter]
%\newtheorem{property}{Property}[chapter]
%\newtheorem{proposition}{Proposition}[chapter]
%\newtheorem{lemma}{Lemma}[chapter]
%\newtheorem{corollary}{Corollary}[chapter]

%\newtheorem*{theorem*}{Theorem}[chapter]
%\newtheorem*{property*}{Property}[chapter]
%\newtheorem*{proposition*}{Proposition}[chapter]
%\newtheorem*{lemma*}{Lemma}[chapter]
%\newtheorem*{corollary*}{Corollary}[chapter]

\theoremstyle{definition}
%\newtheorem{definition}{Definition}[chapter]
%\newtheorem{example}{Example}[chapter]
%\newtheorem*{definition*}{Definition}[chapter]
%\newtheorem*{example*}{Example}[chapter]

\theoremstyle{remark}
%\newtheorem{remark}{Remark}[chapter]
%\newtheorem{notation}{Notation}[chapter]

%& Example of background tag
% \AddToShipoutPicture{%
% \begin{tikzpicture}[remember picture,overlay]
%   \node [rotate=60,scale=10,text opacity=0.1] at (current page.center) {Brouillon};
% \end{tikzpicture}}

%& Example of header tag
% \AddToShipoutPicture{%
% \tikzstyle{block} = [draw, thick, color=blue, scale=1.5,rectangle, minimum height=3em, minimum width=6em]
% \begin{tikzpicture}[remember picture,overlay]
%   \node [coordinate] at (current page.north) (accroche) {};
%   \node [block, below of=accroche] {Diffusion restreinte};
% \end{tikzpicture}}

%& Another example of header tag
% \AddToShipoutPicture{%
% \tikzstyle{block} = [draw, thick, color=red, scale=1.5,rectangle, minimum height=3em, minimum width=6em]
% \begin{tikzpicture}[remember picture,overlay]
%   \node [coordinate] at (current page.north) (accroche) {};
%   \node [block, below of=accroche] {Confidentiel Défense};
% \end{tikzpicture}}

%%% Local Variables:
%%% mode: latex
%%% TeX-master: "../phdthesis"
%%% End:
